\section{State of the art}
%\subsection{Analysis of the market}
This analysis is performed in two important directions: the commercial and the research ones. In the commercial field, only 20\% of the hotel managers are used to looking for consistent demand statistics and use them actively for demand forecasting and profit management operations. The majority consider only the tourist information provided by specialized agencies instead of looking at past booking data to conclude on this. The existence of a software managing the whole bookings system is a common characteristic of many hotels. 

The algorithms existing in the market either in specialized softwares or open source solvers, over all those being used in Italy require the owner of the building to load the prices and priorities of the rooms for making its decisions. These softwares, however, tackle mostly the point of revenue management optimisation (offer campaigns and calculation of optimal room prices) but do not direct its resources in determining the optimal booking out of a given booking request situation. Until today, random assignments and first book - first take policies have been implemented in order to achieve not only a profit but also convenient client satisfaction.

If a few quantity of the software market is directed to optimising the booking, one can imply that even a smaller one attempts to optimize the objective from different perspectives. A smart agent in this sense would be an attractive proposal for optimising decreasing the energy consumption while maintaining the revenue at a maximum level. As commented afterwards, the problem of client satisfaction and personalized use forecasting while ensuring feasibility on the final solutions determines also an important developmental direction and a allows to generate an intelligent framework capable of deciding on the assignment based on a more objective complexity level \cite{grids}.

In the field of research, many works have attempted to manage the consumption of energy from a centralized point of view, in which small clients subscribe to a bigger entity in order to decrease the consumption by means of a closed communication mechanism \cite{central}. Domotics is an area dedicated to the automation of environments and to the increasing of their receptiveness to human beings in order to perform tasks more optimally. One of the most salient topics in the present has been that of providing different frameworks with the awareness of energy consumption and with it generate even communication patterns that ensure optimality in request reception (e.g. turn off the lights and turn them on again)\cite{web}.

A framework like the one proposed in this work would therefore:
\begin{enumerate}
	\item Represent a good opportunity to compete against softwares using other decision techniques.
	\item Ensure the capacity of solving multi objective problems.
	\item In both dimensions it implies improvements with respect to existing techniques
\end{enumerate}


%\subsection{Criteria used}
%????